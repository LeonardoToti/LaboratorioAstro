\section{Analisi Dati e Velocità}
\label{Analisi Dati e Velocità}



\subsection{Lettura dati}
Come vengono letti i dati, immagine tutta colorata prima senza media, poi media fatta sulla scala lineare, con grafico. 
Media mobile, con grafico media mobile


\subsection{Temperatura di Brillanza}
Ora, si vuole calcolare la temperatura di brillana $T_{B}$. Quest'ultima è una grandezza fondamentale nello studio di una sorgente astrofisica. Per ricavarla, possiamo utlizzare l'approssimazione $T_{B} \simeq T_{A}$, dove $T_{A}$ è la temperatura di antenna, poichè ci troviamo nel caso di sorgente estesa, $\Omega_{A} \ll \Omega_{B}$.\\
Per determinare $T_{A}$ utilizziamo la relazione:
\begin{equation}
    W_{out}=GT_{sys}=G[T_{A}+T_{room}]
\end{equation}
dove $G, T_{sys}$ e $T_{room}$ sono rispettivamente il Gain, la temperatura del sistema e la temperatura di rumore.
Da qui appunto si ottiene che:
\begin{equation}
    T_{A}=T_{sys}-T_{rum}=\frac{W_{out}}{G}-T_{room} 
\label{temp antenna}
\end{equation}
Nella temperatura ottenuta però, oltre al termine dovuto al segnale proveniente dal cielo, $T_{cielo}$, è contenuta anche la temperatura di rumore del cavo, $T_{c}$, secondo la relazione:
\begin{equation}
    T_{A}=T_{cielo}e^{-\tau}+T_{c}(1-e^{-\tau})
\end{equation}
Perciò, la temperatura di brillanza del solo segnale è data da:
\begin{equation}
    T_{cielo}=[T_{A}-T_{c}(1-e^{-\tau})]e^{\tau}
\end{equation}

\subsection{Sottrazione rumore}
Quindi, per prima cosa si converte il segnale in temperatura d'antenna utilizzando la relazione \eqref{temp antenna}.\\\\
FIGURA GRAFICO TEMPERATURA DI BRILLANZA PRIMA DELLA RIMOZIONE DEL FONDO E DEL CAVO\\\\
Successivamente si procede alla rimozione del segnale di fondo e della temperatura di rumore del cavo.\\
Per la rimozione del fondo si utilizza la libreria $Specutils$ di Python. Questa permette, tramite la funzione $fit\_continuum$, di fare un fit polinomiale del segnale escludendo determinate regioni. Nel nostro caso abbiamo escluso le regioni dei picchi che sovrastano il fondo. La funzione $fit\_continuum$, in particolare, per eseguire il fit, utilizza il Polinomio di Čebyšëv, la cui definizione in forma esplicita è data da:
\begin{equation}
    T_n(x)=\sum_{h=0}^{[n/2]} (-1)^h {n \choose 2h} x^{n-2h} (1-x^2)^h
\end{equation}
dove con $[n/2]$ si intende la parte intera di $n/2$. RFERENZA A WIKIPEDIA

\subsubsection{Sottrazione contributo cavo}
Immagine da fare del solo segnale senza cavo




\subsubsection{Rimozione del segnale di fondo}
parlare anche di specutils, inserire immagine con fit in arancione, e immagine da fare con segnale rimosso


\subsection{Fit del segnale}
Fit del segnale con la multi gaussiana

\subsection{Calcolo delle velocità dei tre picchi}
Velocità con effetto doppler dei singoli picchi

\subsection{Correzione col moto di rivoluzione terrestre}

\subsubsection{Correzione}
Correzione, grafico andamenti non più cresenti 


\subsubsection{Dispersione delle velocità}
Istogrammi, fit, e dispersione