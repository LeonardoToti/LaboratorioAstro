\section{Analisi Dati e Velocità}
\label{Analisi Dati e Velocità}



\subsection{Lettura dati}
Si considera il segnale osservato dal ricevitore a 1,4 GHz e lo si grafica in funzione della frequenza:
\\\\
grafico con tanti colori
\\\\
Ogni colore rapprsenta i dati di un particolare record. Di conseguenza, il grafico ottenuto è costituito da 150 colori differenti. Le intensità dei record sono molto simili l'una con l'altra.
Il segnale graficato è espresso in scala logaritmica, si passa quindi alla scala lineare convertendo i valori da dBmW a mw. Viene fatta, poi, una media del segnale lineare sui 150 record.\\
Successivamente, si restringe lo studio del segnale nell'intorno della regione della riga H 21 e si osserva che si hanno tre picchi relativi a quest'ultima. Questo è dovuto alla presenza di tre regioni di idrogeno distinte che si muovono a velocità differenti.//  
Si nota che il segnale risulta essere disturbato. Viene quindi eseguita una media mobile, mediando ciascun dato col precedente e col successivo.


\subsection{Temperatura di Brillanza}
Ora, si vuole calcolare la temperatura di brillanza $T_{B}$. Quest'ultima è una grandezza fondamentale nello studio di una sorgente astrofisica. Per ricavarla, possiamo utilizzare l'approssimazione $T_{B} \simeq T_{A}$, dove $T_{A}$ è la temperatura di antenna, poiché ci troviamo nel caso di sorgente estesa, $\Omega_{A} \ll \Omega_{B}$.\\
Per determinare $T_{A}$ utilizziamo la relazione:
\begin{equation}
    W_{out}=GT_{sys}=G[T_{A}+T_{room}]
\end{equation}
dove $G, T_{sys}$ e $T_{room}$ sono rispettivamente il Gain, la temperatura del sistema e la temperatura di rumore.
Da qui appunto si ottiene che:
\begin{equation}
    T_{A}=T_{sys}-T_{rum}=\frac{W_{out}}{G}-T_{room} 
\label{temp antenna}
\end{equation}
Nella temperatura ottenuta però, oltre al termine dovuto al segnale proveniente dal cielo, $T_{cielo}$, è contenuta anche la temperatura di rumore del cavo, $T_{c}$, secondo la relazione:
\begin{equation}
    T_{A}=T_{cielo}e^{-\tau}+T_{c}(1-e^{-\tau})
\end{equation}
Perciò, la temperatura di brillanza del solo segnale è data da:
\begin{equation}
    T_{cielo}=[T_{A}-T_{c}(1-e^{-\tau})]e^{\tau}
\label{temp cielo}
\end{equation}

\subsection{Sottrazione rumore}
Quindi, per prima cosa si converte il segnale in temperatura d'antenna utilizzando la relazione \eqref{temp antenna}.\\\\
FIGURA GRAFICO TEMPERATURA DI BRILLANZA PRIMA DELLA RIMOZIONE DEL FONDO E DEL CAVO\\\\
Successivamente si procede alla rimozione del segnale di fondo e della temperatura di rumore del cavo.\\
Per la rimozione del fondo si utilizza la libreria $Specutils$ di Python. Questa permette, tramite la funzione $fit\_continuum$, di fare un fit polinomiale del segnale escludendo determinate regioni. Nel nostro caso abbiamo escluso le regioni dei picchi che sovrastano il fondo. La funzione $fit\_continuum$, in particolare, per eseguire il fit, utilizza il Polinomio di Čebyšëv, la cui definizione in forma esplicita è data da:
\begin{equation}
    T_n(x)=\sum_{h=0}^{[n/2]} (-1)^h {n \choose 2h} x^{n-2h} (1-x^2)^h
\end{equation}
dove con $[n/2]$ si intende la parte intera di $n/2$. REFERENZA A WIKIPEDIA\\\\GRAFICO FIT SPECUTILS\\\\
Infine, per rimuovere la temèperatura di rumore del cavo e ottenere quindi la temperatura di brillanza della sorgente si utilizza l'equazione \eqref{temp cielo}.\\\\GRAFICO TEMP BRILLANZA FINALE

\subsection{Fit del segnale}
I picchi che osserviamo non sono delle Delta di Dirac, bensì presentano un allargamento dovuto a diversi fenomeni fisici. Inanzitutto, vi è l'allargamento naturale, dovuto al principio di indeterminazione di Heisemberg, per il quale $\Delta E\Delta t\geq\frac{\hbar}{2}$. Inoltre, è anche presente l'allargamento per collissioni, il quale si verifica quando atomi e molecole vengono in contatto con conseguente perturbazione dei loro livelli energetici. Questo dipende dalla loro configurazione elettronica e dalla velocità dell'urto. Infine, un altro tipo di fenomeno tipico è l'allargamento Doppler, causato da una distribuzione di velocità di atomi e molecole. Le diverse velocità delle particelle generano diversi Doppler shift, che sono la causa dell'argamento della riga.\\
Si effettua quindi un fit gaussiano di ciascun picco singolo e, in seguito, un fit multigaussiano dei tre picchi insieme.\\\\
GRAFICO FIT \\\\CONSIDERAZIONE SUI FIT\\\\

\subsection{Calcolo delle velocità dei tre picchi}
Tramite l'effetto Doppler è possibile ricavare la velocità delle tre regioni di emissione. In particolare, l'effetto Doppler consiste nella variazione apparente della frequenza percepita da un osservatore, rispetto al valore vero della frequenza emessa da una sorgente in moto rispetto all'osservatore stesso. Vale quindi:
\begin{equation}
    f=\frac{c}{c\pm v}f_{0}
\end{equation}
Dove $f$ è la frequenza percepita dall'osservatore, $f_{0}$ è la frequenza della sorgente e $v$ è la velocità della sorgente. Al denominatore si usa il più quando la sorgente si allontana dall'osservatore, mentre si usa il meno quando si avvicina. Quest'ultimo è il nostro caso. Perciò la velocità delle tre regioni di emissione si ricava dalla legge:
\begin{equation}
    v=c (1-\frac{f_{0}}{f})
\end{equation}
Tramite il fit si ricava la frequenza dei tre picchi e quindi le velocità.

\subsection{Correzione col moto di rivoluzione terrestre}

\subsubsection{Correzione}
Correzione, grafico andamenti non più cresenti 


\subsubsection{Dispersione delle velocità}
Istogrammi, fit, e dispersione