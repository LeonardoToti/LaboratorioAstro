\section{Introduzione}

\subsection{Obiettivi}
Lo scopo dell'esperienza è lo studio della regione del Cigno, con particolare attenzione all'analisi dello spettro di emissione dell'idrogeno neutro $\textbf{HI}$, andando a soffermarci sulla riga 21 cm. 
In ultima analisi verrà compiuta una mappa della regione di cielo osservata, per evidenziare la distribuzione spaziale dell'idrogeno.
\\\\
Per condurre tale esperimento sarà utilizzata la strumentazione universitaria, parabola e ricevitore, opportunamente calibrata.
\\\\
Di seguito verrà esposta una breve trattazione teorica, a cui seguirà una descrizione accurata dell'apparato strumentale. Passando poi a descrivere la calibrazione della parabola e del ricevitore, la quale è necessaria per l'effettiva analisi dati e mappatura della regione. Infine riassumeremo i risultati ottenuti nella conclusione.



\subsection{Sorgente}
La sorgente galattica di interesse è il Cigno, individuabile attraverso le coordinate celesti, ascensione retta e declinazione, pari a RA $\sim$ 308 deg e Dec $\sim$ 42 deg.
Studiando il flusso, ovvero radiazione di natura elettromagnetica, rilevato in funzione della frequenza si ottiene lo spettro caratteristico della sorgente. Quest'ultimo è ascrivibile a tre componenti distinte: 
\begin{itemize}
\item Spettro di emissione, spettro discreto con linee di emissione a determinate frequenze;
\item Spettro d'assorbimento, spettro discreto con linee d'assorbimento a determinate frequenze;
\item Spettro continuo causato dall'emissione a tutte le frequenze, dovuto al comportamento della sorgente assimilabile a un corpo nero a una data temperatura. 
\end{itemize}

\subsubsection*{Spettro di emissione}
L'origine dello spettro di emissione è il fenomeno dell'eccitamento e il conseguente diseccitamento degli elettroni negli atomi. La radiazione osservata è l'emissione di fotoni con un'energia prossima al salto energetico compiuto.

%{\includegraphics[scale=0.10]{transizione.pdf}}
%sistemare immagine, mettere la didascalia e riferila al testo(?)

\subsubsection*{Spettro di assorbimento}
Lo spettro di assorbimento è imputabile a un mezzo, solitamente a temperatura minore della sorgente, interposto tra essa e l'osservatore. I fotoni emessi dalla sorgente primaria eccitano atomi nel mezzo, a loro volta diseccitandosi emettono fotoni casualmente in ogni direzione, dando luogo all'assenza di righe nello spettro.

\subsubsection*{Spettro continuo}
A generare lo spettro continuo concorrono due principali fenomeni: 
\begin{itemize}
\item Radiazione termica, a sua volta distinguibile in bremsstrahlung, corpo nero e polvere interstellare;
\item Radiazione non termica, emissione di sincrotone. 
\end{itemize}

In particolar modo, il fenomeno del bremsstrahlung è dovuto al rallentamento di un elettrone libero all'interno dell'atomo, essendo una particella carica in decelerazione emette fotoni in modo continuo. Particolarmente diffuso in regione con HII, ovvero nubi di idrogeno ionizzato da raggi ultravioletti associati a stelle nascenti o morenti.
\\\\
La nube di idrogeno è a una data temperatura, per tale motivo emette radiazione con uno spettro assimilabile a un copro nero, piccando a una sua temperatura caratteristica. 
\\\\
La polvere interstellare assorbe luce stellare in banda ottica, aumento la propria temperatura. Tale aumento comporta emissione termica in banda infrarossa.
\\\\
Per quanto concerne la radiazione di sincrotone, essa è dovuta all'emissione di fotoni da parte di elettroni in moto variabile all'interno del campo magnetico galattico. Lo spettro è tipico di un corpo grigio.


\subsection{Riga idrogeno}
Il soggetto principale dello studio è la riga a 21 cm dell'idrogeno neutro. Prende tale nome dalla lunghezza d'onda a cui si manifesta 21,10611405413 cm, corrispondenti a 1420,405 MHz, nella banda delle microonde. La transizione avviene tra due stati iperfini, come riportato in figura $\textit{figuraaaaaaaaa 1 e 2 di wikipedia}$, a causa di uno spin flipping, ovvero transizione da uno stato in cui $e^{-}$ e $p^{+}$ hanno spin parallelo ad uno stato in cui $e^{-}$ e $p^{+}$ hanno spin antiparallelo. La struttura iperfine si ottiene considerando i termini di interazione tra: il momento magnetico del protone e il campo generato dall'elettrone, dipolo-dipolo magnetico tra $e^{-}$ e $p^{+}$, e momento magnetico di $e^{-}$ col campo magnetico interno al $p^{+}$.
Il decadimento è estremamente proibito e il tempo di vita medio per tale stato eccitato è dell'ordine di $\tau \simeq 10 ^{7}$  anni.
Nonostante tale caratteristica è possibile osservare la transizione  a causa dell'elevato numero di atomi di H nella regione di analisi, pur considerando una densità bassa pari a $\simeq$ 100 $\frac{atomi}{cm^{3}}$. $\textit{mettere la referenza a cui fa riferimento wikièedia}$



