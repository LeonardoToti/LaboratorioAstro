\section{Calibrazione}
La calibrazione del ricevitore a 1.4 a GHz viene effettuata utilizzando due sorgenti di riferimento, che sono due carichi coassiali adattati, cioè due resistenze a 50$\Omega$. Questi lavorano a due temperature diverse: uno a temperatura ambiente (\textit{warm load}) e l'altro alla temperatura di ebollizione dell'azoto liquido, ovvero 77,36K (\textit{cold load}). \\\\Il carico a temperatura criogenica è connesso al ricevitore tramite un cavo coassiale il quale è immerso parzialmente nell'azoto liquido. Perciò, dato che la temperatura del cavo non è uniforme, ne vanno studiate le caratteristiche di attenuazione in laboratorio, su un cavo analogo, a temperatura ambiente ed a temperatura criogenica. Per farlo si utilizza un analizzatore vettoriale di reti (VNA).

\subsection{Attenuazione cavo coassiale}
\subsubsection{VNA e matrice di scattering}
Il VNA è una macchina che permette di misurare le proprietà di trasmissione di un cavo coassiale.
\subsubsection{Matrice di scattering con separatori termici}
\subsubsection{Matrice di scattering con separatori termici e cavo di rame}
\subsubsection{Matrice di scattering del cavo di rame}
\subsubsection{Calcolo del coefficiente di attenuazione}


\subsection{Calibrazione del ricevitore a 1.4 GHz}
La conoscenza del profilo del coefficente di attenuazione in funzione della temperatura permette di attuare la calibrazione del ricevitore a 1,4 GHz. Si consideri come warm load e cold load, un cavo rispettivamente di lunghezza 12 m e 120 cm. Il cold load è immerso parzialmente nell'azoto liquido portando ad un'attenuazione del segnale.\\
Per ottenere il segnale effettivo letto dal ricevitore è quindi necessario considerare la variazione della temperatura all'interno del cavo.

\subsubsection{Profilo di temperatura del cavo}

La stima del profilo di temperatura è resa possibile grazie a 4 sensori criogenici e 2 sensori a temperatura ambiente posizionati sul cavo (come mostra la figura sottostante). 
\\
figura dei sensori con il metro.
\\
I primi 4 sensori restituiscono il valore in Kelvin, mentre gli ultimi due attuano la loro misurazione in gradi Celcius.\\
Successivamente all' immersione completa del cold load nel criostato che contiene Azoto liquido a pressione ambiente. Dai valori ottenuti dai sensori è quindi possibile determinare l'andamento della temperatura in funzione della posizione. 
\\\\
Tabella file temperatura in funzione posizione
\\\\
Infine,si fittino i seguenti valori con una funzione sigmoide
\begin{equation}
T= \dfrac{a}{1+e^{-k(x-x_{0})}}
\end{equation}
Il grafico che si ottiene è il seguente:
\\\\
grafico andamento temperatura
\\\\
Successivamente, si ricavi dal profilo di  temperatura e dall'andamento del coefficiente di attenuazione in funzione della temperatura il valore di temperatura del cavo cold load e warm load.
\subsubsection{Temperatura del cavo cold load}
L'immersione parziale del cavo cold load all'interno del criostato comporta che la temperatura del carico criogenico deve essere corretta. La propagazione della temperatura nel cavo segue il seguente andamento:
\begin{equation}
T_{b}= T_{s}e^{-\tau} + T_{C}(1-e^{-\tau})
\label{Formula ricorsiva}
\end{equation}
dove $T_{s}$ è pari a 77,36 K, ovvero la temperatura di ebolizione dell'azoto liquido, $T_{c}$ è il valore della temperatura studiata nel paragrafo [?] e $\tau$ è il coefficiente di attenuazione studiato nel paragrafo [?].
Si applichi tale relazione iterativamente a dei tratti di lunghezza $\Delta$x. Si scelga come step di lunghezza un valore tale da poter considerare il tratto di cavo preso in considerazione isotermo. Si scelga quindi un valore di $\Delta$x = 1mm.
L'iterazione dell'equazione \eqref{Formula ricorsiva} è dovuta al fatto che il sistema risulta essere in equilibrio per tutto il tratto di cavo immerso nell'azoto liquido. \'E quindi possibile considerare valida la relazione $T_{b} = T_{s} = 77,36 K$. Nel tratto di cavo scoperto il sistema non è più all'equilibrio, la temperatura del cavo varia ed è quindi necessario determinare il coefficiente di attenuazione $\tau$ del cavo.\\
Successivamente a tali considerazioni si ricava un valore di $T_{b}$ = ... .
%\cite{Cold load:Cold load}.

\subsubsection{Temperatura del cavo warm load}\label{Temperatura del cavo warm load}
Per determinare la temperatura del cavo warm load il procedimento è simile a quello utilizzato nel paragrafo \ref{Temperatura del cavo cold load} con la differenza che la totalità del cavo è a temperatura ambiente. Quindi la sorgente e il cavo hanno circa la stessa temperatura $T_{s}$ $\sim$ $T_{c}$ $\sim$ 273,15 K.\\
Di conseguenza risulta possibile semplificare l'equazzione \eqref{Formula ricorsiva} nel seguente modo:
(da riguardare la formula perchè quella della relazione dell'amica di Simo è sbagliata.)
\subsubsection{Guadagno del ricevitore}
\subsubsection{Temperatura di rumore}
