\section{Calibrazione}
La calibrazione del ricevitore a 1.4 a GHz viene effettuata utilizzando due sorgenti di riferimento, che sono due carichi coassiali adattati, cioè due resistenze a 50$\Omega$. Questi lavorano a due temperature diverse: uno a temperatura ambiente (\textit{warm load}) e l'altro alla temperatura di ebollizione dell'azoto liquido, ovvero 77,36K (\textit{cold load}). \\\\Il carico a temperatura criogenica è connesso al ricevitore tramite un cavo coassiale il quale è immerso parzialmente nell'azoto liquido. Perciò, dato che la temperatura del cavo non è uniforme, ne vanno studiate le caratteristiche di attenuazione in laboratorio, su un cavo analogo, a temperatura ambiente ed a temperatura criogenica. Per farlo si utilizza un analizzatore vettoriale di reti (VNA).

\subsection{Attenuazione cavo coassiale}
\subsubsection{VNA e matrice di scattering}
Il VNA è una macchina che permette di misurare le proprietà di trasmissione di un cavo coassiale.
\subsubsection{Matrice di scattering con separatori termici}
\subsubsection{Matrice di scattering con separatori termici e cavo di rame}
\subsubsection{Matrice di scattering del cavo di rame}
\subsubsection{Calcolo del coefficiente di attenuazione}


\subsection{Calibrazione del ricevitore a 1.4 GHz}
\subsubsection{Profilo di temperatura del cavo}
\subsubsection{Temperatura del cavo cold load}
\subsubsection{Temperatura del cavo warm load}
\subsubsection{Guadagno del ricevitore}
\subsubsection{Temperatura di rumore}